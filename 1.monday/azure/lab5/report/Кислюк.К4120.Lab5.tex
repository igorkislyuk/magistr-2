\documentclass[14pt,a4paper]{extreport}

% Caption configuration
\usepackage{caption}

% Times New Roman, be sure to build with XeLaTeX
\usepackage{fontspec}
\usepackage[russian]{babel}
\setmainfont{Times New Roman}

% mock data
\usepackage{lipsum}

% russian
\usepackage[utf8]{inputenc}

\linespread{1.5}

\usepackage{graphicx}
\usepackage[
    letterpaper,
    left        = 2cm,
    right       = 1cm,
    top         = 1cm,
    headheight  = 2cm
]{geometry}

%% New commands

% caption | screenshot index
\newcommand{\screenshot}[2]{\begin{figure}[ht]%
\centering\includegraphics[width=0.8\textwidth]{../Screenshots/#2}%
\caption{#1}%
\label{picture#2}%
\end{figure}%
}

\newcommand{\header}[1]{%
{
\fontsize{16pt}{14pt}\selectfont
\begin{center}
\textbf{\MakeUppercase{#1}:}
\end{center}
}
}

\newcommand{\prepod}{Ананченко~И.~В.}
\newcommand{\igork}{Кислюк~И.~В.}

% Configurations

% Рис 1. -> Рис 1 --, Таблица 1. -> Таблица 1 --
% Рис 1. -> Рисунок 1
\DeclareCaptionFormat{myformat}{\fontsize{12}{12}\selectfont#1#2#3}
\captionsetup[figure]{format={myformat},name={Рисунок},labelsep=endash}

\begin{document}

	\begin{titlepage}
	\begin{center}	
		\fontsize{14pt}{14pt}\selectfont
		МИНИСТЕРСТВО ОБРАЗОВАНИЯ И НАУКИ\\

		\vspace*{0.6\baselineskip}
		
		САНКТ-ПЕТЕРБУРГСКИЙ НАЦИОНАЛЬНЫЙ ИССЛЕДОВАТЕЛЬСКИЙ УНИВЕРСИТЕТ ИНФОРМАЦИОННЫХ ТЕХНОЛОГИЙ, МЕХАНИКИ И ОПТИКИ
		
		\vspace*{0.6\baselineskip}
		ФАКУЛЬТЕТ ИНФОКОММУНИКАЦИОННЫХ ТЕХНОЛОГИЙ
		КАФЕДРА ПРОГРАММНЫХ СИСТЕМ
	
		\vspace*{7\baselineskip}
		\fontseries{m}\fontsize{19pt}{18pt}\selectfont
		Отчет по лабораторной работе
		
		\fontseries{m}\fontsize{20pt}{18pt}\selectfont
		\textbf{Create and configure virtual networks}\\
		\vspace*{1.15\baselineskip}
		\end{center}
	
	\vspace*{2\baselineskip}
	\begin{flushright}
	\fontseries{m}\fontsize{14pt}{14pt}\selectfont
	\textbf{Выполнил:}\\
	\igork\\
	студент группы К4120\\
	Проверил: \prepod\\
	\end{flushright}
	
	\vspace{\fill}
	\begin{center}
	Санкт-Петербург\\
	2018
	\end{center}
	
\end{titlepage}

\newpage

\header{Цель работы}

\fontsize{14pt}{14pt}\selectfont

Ознакомиться с облачными технологиями и возможностями, которые предоставляет компания Microsoft
\clearpage

\header{Ход работы}

\begin{enumerate}
\item Creating virtual networks
\item Verifying virtual network functionality
\end{enumerate}

\screenshot{Создание первой сети}{1}
\screenshot{Создание второй сети}{2}
\screenshot{Добавление первого пиринга}{3}
\screenshot{Добавление второго пиринга}{4}
\screenshot{Создание виртуальной машины}{5}
\screenshot{Настройка сети виртуальной машины}{6}
\screenshot{Создание второй виртуальной машины}{7}
\screenshot{Настройка сети второй виртуальной машины}{8}
\screenshot{Подключение к виртуальной машине}{9}
\screenshot{Запущенная виртуальная машина}{10}
\screenshot{Получение сетевого адреса}{11}
\screenshot{Подключение по приватному адресу}{12}
\screenshot{Ввод аутентификационных данных}{13}
\screenshot{Подключенная виртуальная машина}{14}
\screenshot{Удаление групп ресурсов}{15}

\clearpage
\header{Вывод}

Было проведено ознакомление с технологиями Azure

\end{document}

